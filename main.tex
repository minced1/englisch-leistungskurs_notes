% arara: lualatex: {options: []}
\documentclass[10pt]{article}
\usepackage{notestex,csquotes,tabularx}
\SetColorProfile{9C8C52}{BEA035}{3D4CDB}{525786}{5C5748}
\usepackage[english]{babel}
\renewcommand{\arraystretch}{1.5}

\begin{document}
\title{{Englisch}\\{\normalsize{\itshape Leistungskurs}}}
	\author{Never Gude}
	\affiliation{
	Carl-Friedrich-Gauß Gymnasium Hockenheim\\
	\href{https://github.com/minced1}{GitHub}\\
	}
	\emailAdd{j.gude@posteo.de}
	\license{This work is licensed under \href{https://creativecommons.org/licenses/by-nc-sa/4.0/}{CC BY-NC-SA 4.0}}{\ccbyncsa}
	\maketitle
	\newpage
	\pagestyle{fancynotes}

\part{Kommunikationsprüfung}
	\label{part:kompruef}
\section{United Kingdom}
	\label{sec:uk}
\subsection{Empire and Commonwealth}
	\label{ssec:uk@empire}
\subsection{Immigration}
	\label{ssec:uk@immig}
\subsection{Global role: EU and Brexit}
	\label{ssec:uk@brexit}

\section{India}
	\label{sec:india}
\subsection{Partition of British India}
	\label{ssec:india@britindia}
\subsection{Poverty}
	\label{ssec:india@poverty}
\subsection{Language}
	\label{ssec:india@language}
\subsection{Technology vs. Backwardsness}
	\label{ssec:india@tech}
\newpage
\section{Globalisation}
	\label{sec:global}
The focus of globalisation has changed over the years. As it progressed, the world shrunk.
People grew more connected and technology evolved.
\begin{table}[h!]
	\centering
	\begin{tabularx}{\textwidth}{ >{\raggedleft}X | X X X}
		\hline
		& {\sffamily\bfseries 1.0} & {\sffamily\bfseries 2.0} & {\sffamily\bfseries 3.0} \\ \hline
		{\sffamily\bfseries time frame} & 1492--1800 & 1800--2000 & 2000--today \\
		{\sffamily\bfseries size of world} & medium & small & tiny \\
		{\sffamily\bfseries major players} & countries & companies & individuals \\
		{\sffamily\bfseries driving force} & countries, governments, (muscle) power & multinational companies & more diverse (non-western, non-white), empowerment through digitalisation \\
		{\sffamily\bfseries key technology} & collaboration, trade, sailing ships & railway, steam engine & digital technology, fiber-optic cable, internet, satellites \\
		{\sffamily\bfseries primary question} & Where does country fit in and go global? & Where does company fit in and go global? & Where do I fit in and go global? \\ \hline

	\end{tabularx}
	\caption{Change of globalisation over the years}
	\label{tab:my_label}
\end{table}

\subsection{Social aspects}
	\label{ssec:global@social}
\subsection{Political aspects}
	\label{ssec:global@political}
\subsection{Economic aspects}
	\label{ssec:global@economic}

\section{United States of America}
	\label{sec:usa}
\subsection{History}
	\label{ssec:usa@history}
\begin{itemize}[leftmargin=5pt+\widthof{1775 - 1783}]
\item[1607] First permanent English settlement at Jamestown (Virginia)
\item[1620] 'Pilgrim Fathers' (i.e. English Puritans) found colony at Plymouth (Massachusetts)
\item[1775 - 1783] 13 colonies fight the War of Independence against Britain
\item[4.7.1775] The Declaration of Independence is signed
\item[1789] US Constitution goes into effect. The former colonies are now the United States of America
\item[1803] The Louisiana Purchase: the United States doubles in size
\item[1861 - 1865] Civil War; slavery ends in the South
\item[1890] The last major battle against Native American tribes. Congress officially declares the frontier closed
\item[1917] USA enters World War I (first involvement in a European conflict)
\item[1919] Women are given the right to vote
\item[1919] Great depression begins
\item[1941] USA enters World War II following Japanese attack on Pearl Harbor
\item[1945] USA drops atomic bombs on Hiroshima and Nagasaki
\item[1945] Cold War begins as USA and Soviet Union emerge as new world powers
\item[1954] US Supreme Court declares segregated schools illegal
\item[1955] Civil Rights Movement begins with the Montgomery (Alabama) bus boycott
\item[1989 - 1991] Cold War ends with the collapse of the Soviet Union
\item[2001] Terrorist attacks on the Pentagon and the World Trade Center
\item[2006] US population reaches 300 million
\item[2008] Financial crisis begins
\item[2008] Barack Obama is elected US President. He is the first African American holding the office (2009 - 2017)
\item[2017] Donald Trump becomes the US President. His slogan is 'America First'
\item[2020] Storm of the Capitol
\item[2021] Joe Biden is elected US President
\end{itemize}
\subsection{Constitution}
	\label{ssec:usa@constitution}
\paragraph{The Declaration of Independence}
In the Declaration of Independence the thirteen American colonies announced their freedom from
British rule. One July 11th 1776, Congress appointed a committee to draft a formal Declaration of
Independence. Thomas Jefferson, one of the appointed members, was asked to write the draft,
which he completed in about two weeks. The date of its adoption by the Second Continental
Congress on July 4th 1776, is celebrated as the birthday of the United States.
\begin{example}
When, in the Course of human events, it becomes necessary for one people to
dissolve the political bands which have connected them with another, and to
assume among the powers of the earth, the separate and equal station to which
the Laws of Nature and of Nature's God entitle them, a decent respect to the
opinions of mankind requires that they should declare the causes which impel
them to the separation.\\
We hold these truths to be self-evident, that all men are created equal, that they
are endowed by their Creator with certain unalienable Rights, that among these are
Life, Liberty and the pursuit of Happiness.--That to secure these rights,
Governments are instituted among Men, deriving their just powers from the
consent of the governed, --That whenever any Form of Government becomes
destructive of these ends, it is the Right of the People to alter or to abolish it, and
to institute new Government, laying its foundation on such principles and
organizing its powers in such form, as to them shall seem most likely to effect their
Safety and Happiness.
\end{example}

\paragraph{The Preamble to the United States Constitution}
The Preamble to the United States Constitution, beginning with the words \enquote{We the People}, is a brief
introductory statement of the Constitution's fundamental purposes and guiding principles. Courts
have referred to it as reliable evidence of the Founding Fathers' intentions regarding the
Constitution's meaning and what they hoped the Constitution would achieve.
\begin{example}
We the People of the United States, in Order to form a more perfect Union,
establish Justice, insure domestic Tranquility, provide for the common defence,
promote the general Welfare, and secure the Blessings of Liberty to ourselves and
our Posterity, do ordain and establish this Constitution for the United States of
America.
\end{example}

\paragraph{The Bill of Rights}
The original Constitution of the USA was concerned with the way the new government would be
structured and what powers it had. People were worried that the government might become too
powerful and autocratic, just like the British king, so a Bill of Rights consisting of ten amendments
(i.e. changes) to the Constitution was quickly passed in 1791. These amendments were intended to
protect the people from their rulers. The Bill of Rights describes the basic rights of the people and
forbids the government from denying these liberties. Included are the freedoms of speech, religion,
the press, and the right to assemble. Today the Bill of Rights also serves to protect minorities and
individuals from the majority. Altogether there have been 27 amendments to the Constitution.
\begin{example}
\begin{minipage}{\linewidth}
\paragraph{Amendment I}
Congress shall make no law respecting an establishment of religion, or prohibiting the free exercise thereof; or abridging
the freedom of speech, or of the press; or the right of the people peaceably to assemble, and to petition the government
for a redress of grievances.

\paragraph{Amendment II}
A well-regulated militia, being necessary to the security of a free state, the right of the people to keep and bear arms, shall
not be infringed.

\paragraph{Amendment III}
No soldier shall, in time of peace be quartered in any house, without the consent of the owner, nor in time of war, but in a
manner to be prescribed by law.

\paragraph{Amendment IV}
The right of the people to be secure in their persons, houses, papers, and effects, against unreasonable searches and
seizures, shall not be violated, and no warrants shall issue, but upon probable cause, supported by oath or affirmation, and
particularly describing the place to be searched, and the persons or things to be seized.
\end{minipage}
\end{example}
\subsection{Values}
	\label{ssec:usa@values}
\paragraph{Democracy}
Representative government and the separation of powers in a system of checks
and balances are the basis of American Democracy, which should represent American
citizen's interests and be an example to the world.

\paragraph{The Pursuit of Happiness}
The American Constitution sees it as an unalienable right for every citizen to seek
self-fulfilment. The Pursuit of Happiness, together with the right to life and liberty, are at the core of the
US American understanding of democracy.

\paragraph{Individual Freedom}
Americans' understanding of Individual Freedom is shaped by the Founding Fathers' belief that all
people are equal. Through the US Constitution's Bill of Rights, the government is
responsible for protecting each individual's basic \enquote{inalienable} rights among
which are freedom of speech, press, and religion.

\paragraph{Individualism and Self-Reliance}
Closely connected to the notion of freedom, the idea of Individualism and Self-Reliance encourages American
citizens to trust in themselves, their abilities, and their conscience to cope with
their individual fate.

\paragraph{Can-Do-Spirit}
Inventiveness and a tendency to idealize whatever works or is practical illustrate
an American's Can-Do-Spirit which is often attributed to the frontier experience where man
was left to his own devices when it came to survive in the wilderness.

\paragraph{Equal Opportunities}
Over centuries, the concept of the American Dream has taught Americans that
self-realization can be achieved through hard work, thrift, family loyalty, and faith
in free enterprise. Equal Opportunities, however, have not been within reach for all: segregation and
discrimination have interfered with indigenous peoples and minorities' attempts
to realize the rights and freedoms, which are also guaranteed for them by the
Constitution.

\paragraph{Mobility}
For the US nation of immigrants, in which moving elsewhere and making a fresh
start is seen as a practical solution to many a challenge, Mobility can be described as a
form of realized optimism promising a better life ahead.

\paragraph{Patriotism}
Slogans like Proud to be American, the playing of the national anthem at sporting
events or the pledging of allegiance, which starts off every school morning, are
signs of a heartfelt omnipresent Patriotism which show Americans' intensified sense of
national identity and pride in their nation's unique promises.

\paragraph{Progress}
Americans associate Progress with the idea of making use of the opportunities, which
their nation holds in store for them. Hard work and the willingness to make sacrifices
for a better tomorrow promise self-realization as well as an increase in
material plenty from generation to generation.

\paragraph{National Security}
In recent years National Security has come into focus. 9/11 left Americans in a state of shock
and has made certain rights negotiable (right to privacy) which were formerly
undisputed.
\subsection{Immigration}
	\label{ssec:usa@immig}
The English have been going to North America from the late 16th century on; Spain sent people to the
southern part of the region and many Dutch and Germans also went over.

When the U.S. became independent, it was written into the \emph{Constitution} that there could be no
limits on immigration until 1808. The main period of immigration was between 1800 and 1917. Early
in this period, many immigrants arrived from Britain and Germany, and many Chinese went to
California. Later, the main groups were Italians, Irish, Eastern Europeans and Scandinavians. Many
Jews came from Germany and Eastern Europe. Just before World War I, there were nearly a million
immigrants a year. Many immigrants came to New York and Boston, and Ellis Island, near New York,
became famous as a receiving station. The \emph{Immigration Act of 1917}, and other laws that followed it,
limited the number of immigrants and the countries that they could come from. Since then,
immigration has been limited to a few people who are selected for an immigrant visa, commonly
called green card. Hispanics and Asians now make up the largest group of immigrants.

Immigration policy, including illegal immigration to the United States, was a signature issue of former
U.S. president \emph{Donald Trump's} presidential campaign. He repeatedly said that illegal immigrants are
criminals. A hallmark promise of his campaign was to build a substantial wall on the United States-
Mexico border and to force Mexico to pay for the wall.

\begin{table}[h!]
	\centering
	\begin{tabularx}{\textwidth}{l|l}
		\hline
		{\sffamily\bfseries winners} & {\sffamily\bfseries reasons}\\
		employers in industries (service, tourism) & cheap labour \\
		consumers & cheap prices \\
		immigrants children & free education \\
		federal government & ... \\ \hline
		{\sffamily\bfseries losers} & \ {\sffamily\bfseries reasons}\\
		tax paying (immigrants) & ... \\
		local workers & ... \\ \hline
		{\sffamily\bfseries unclear} & {\sffamily\bfseries reasons}\\
		illegal immigrants & ... \\
		community & ... \\ \hline
	\end{tabularx}
	\caption{Winners and losers of illegal immigration}
	\label{tab:my_label}
\end{table}

\subsection{American dream}
	\label{ssec:usa@dream}
\subsection{Racial issues}
	\label{ssec:usa@racial}
\begin{itemize}[leftmargin=5pt+\widthof{1950s and 1960s}]
\item[Since 1640] Slavery is an essential part of cotton and tobacco-growing industries in the South
\item[1775] American War of Independence between Britain and its American colonies
begins
\item[1776] Declaration of Independence adopted
\item[1783] American colonies win the War of Independence against Britain
\item[1860] President Abraham Lincoln tries to abolish slavery; Southern States of America try to leave the Union in protest
\item[1861] Northern States of America move on the South to prevent the break-up of the USA; the American Civil War begins
\item[1863] Emancipation Proclamation signed by Abraham Lincoln, freeing all slaves
\item[1865] American Civil War ends with a victory of the Northern States; 13th Amendment to the Constitution makes slavery illegal in the USA
\item[1865--1877] Reconstruction era; remaking of the Southern United States after the Civil War
\item[1865/1866] The Black Codes, a set of rules, are passed in the South to \enquote{restore all of slavery but its name} southern blacks are
\begin{itemize}
\item denied the right to vote
\item excluded from certain jobs
\item denied the right to own land
\item prohibited from possessing firearms
\end{itemize}
\item[1866] The Ku Klux Klan is founded, resulting in lynching of and discrimination against blacks well into the 20th century
\item[1870] 5th Amendment to the Constitution gives all men the right to vote, regardless of race, but Jim Crow Laws make this difficult in reality and segregation of blacks is still the norm in American society
\item[1880s--1920] peak years of black lynchings
\item[1896] Supreme Court rules, in Plessy v. Ferguson, that so-called \enquote{separate but equal} facilities -- including public transport and schools -- are constitutional
\item[WWI] Segregated regiments of white and African-Americans fight for the U.S.
\item[1916--1970] the Great Migration; the movement of six million African Americans out of the rural Southern United States to the urban Northeast, Midwest and West
\item[WWII] ca. 1 mio African-American soldiers fight for the U.S.
\item[1950s and 1960s] Civil Rights Movement in America, led by Dr Martin Luther King Jr.
\item[1955] Rosa Parks refuses to give up her seat on a bus -- Montgomery Bus Boycott begins
\item[1956] Montgomery Bus Boycott ends - buses are no longer segregated
\item[1957] Little Rock Nine escorted into Little Rock High School by federal troopers, signaling the end of segregated education
\item[1963] Martin Luther King leads a march on Washington and delivers the speech \enquote{I Have a Dream}
\item[1960s] Malcolm X becomes famous leader of Black Muslims; promotion of a separate black state and acceptance of violence as a means of self-defense
\item[1964] Civil Rights Act passed; last of the Jim Crow Laws regarding segregation of Blacks are abolished
\item[1965] Assassination of Malcom X; President Lyndon B. Johnson signs Voting Rights Act; literacy tests required to be allowed to vote are suspended in order to allow many illiterate southern blacks to vote
\item[1968] Martin Luther King assassinated in Memphis, Tennessee
\item[2008] Barack Obama elected as President of the USA
\end{itemize}
\subsection{Gun laws}
	\label{ssec:usa@guns}
\paragraph{Gun culture in the US}
According to reliable estimates there are about 393 million privately owned firearms in
the USA. Around 40\% of Americans live in a household with a gun, which means that
many gun-owners own more than one gun. The USA is the country with the highest gun
ownership rate in the world, followed by Yemen and Switzerland.

Guns have played a major role in American history, especially in the settlement of the
continent and in the various wars that have been fought by Americans. As a means of
self-protection and as a tool for hunting guns were vital to American settlers. In
American popular culture, guns have often been depicted in a positive light. Just think of
Western movies or the characters played by Arnold Schwarzenegger. At the same time
gun ownership has for a long time been a hotly debated topic, each school shooting
serving as a trigger of another round of discussions.

Supporters of gun ownership such as the National Rifle Association (NRA) point to the
2nd Amendment to the Constitution and consider every attempt of the government to
restrict gun ownership as an encroachment on their personal freedom. Claiming that
hundreds of people use guns each year for self-defense, they believe that privately
owned guns reduce crime rate rather than causing it. Opponents of gun ownership, on
the other hand, point to studies that show a decline in homicides and suicides by
firearms after the introduction of stricter gun laws. In addition they maintain that guns in
households can easily lead to accidents or an escalation that wouldn't have happened if
a gun hadn't been involved.

\section{Modern Media}
	\label{sec:media}
\subsection{Modern digital media}
	\label{ssec:media@digital}
\paragraph{School and work} Digital media has been used in business for a long time but
has only recently made its way into school. A lot has changed since then.
\begin{table}[h!]
	\centering
	\begin{tabularx}{\textwidth}{X X}
		\hline
		{\sffamily\bfseries Opportunities} & {\sffamily\bfseries Challenges} \\ \hline
		Easy access to material: Work saved in a cloud storage, research on the internet & Depends on WIFI \\
		Saves paper and weight (books and work accessible with one device) & Uses electrical power, costly in purchase\\
		Sharing information and material quickly & Distraction from class \\ \hline
	\end{tabularx}
	\caption{Opportunities and challenges of digital media in school and work}
	\label{tab:my_label}
\end{table}
\paragraph{Digital image processing} Software like Photoshop has been a huge advancement
in image manipulation. Never have there been such vast options of editing images. Of course
this also has come with new problems like Deep Fakes being indistinguishable from the real person.
\begin{table}[h!]
	\centering
	\begin{tabularx}{\textwidth}{X X}
		\hline
		{\sffamily\bfseries Opportunities} & {\sffamily\bfseries Challenges} \\ \hline
		New tool for expressing creativity & Older techniques being forgotten or devalued \\
		Indistinguishable from real life (Photo realism, Deep Fakes) & Deep fakes can pose a threat to reputation and peace\\
		Cheaper production costs & People losing their jobs \\ \hline
	\end{tabularx}
	\caption{Opportunities and challenges of digital image processing}
	\label{tab:my_label}
\end{table}
\paragraph{Surveillance} The most prominent case of abuse of surveillance systems is China.
In connection with their social credit system, they lead their population to denunciate people
who disagree with the government. Nevertheless, used in the right context and with the right restrictions,
surveillance can help create a safer society.
\begin{table}[h!]
	\centering
	\begin{tabularx}{\textwidth}{X X}
		\hline
		{\sffamily\bfseries Opportunities} & {\sffamily\bfseries Challenges} \\ \hline
		Prevent crime before it happens & Potential for censorship and violation of privacy guidelines \\
		Identification of criminals (CCTV) or proof of innocence & Denunciation when connected with social credit system \\
		Provide jobs: monitor CCTV, hacking criminal group chats & Can be expensive \\ \hline
	\end{tabularx}
	\caption{Opportunities and challenges of digital surveillance}
	\label{tab:my_label}
\end{table}

\subsection{Web 2.0: opportunities and challenges}
	\label{ssec:media@web2}
The Web 2.0 is different from the Web 1.0 in that people can now use portals like YouTube, Instagram
or TikTok to put their content out to the world. Previously, a person had to have personal
Website or Blog. Maintaining that is a rather tedious task, so enterprises like Google and Meta where
founded to take on the job of hosting the servers.

The audience of content creators grew as many people where connected
on a single platform and not spread over various Blogs. But now that the enterprises hosted their users data, they
technically own it if not specified differently in the terms of use.

Thus, recently there
have been new projects trying the build a federated network (called the Web 3.0), which still promises far reach
through a common portal but with the option of private data ownership through privately hosted
servers.
\begin{table}[h!]
	\centering
	\begin{tabularx}{\textwidth}{X X}
		\hline
		{\sffamily\bfseries Opportunities} & {\sffamily\bfseries Challenges} \\ \hline
		Fast spreading of information & Fast spreading of misinformation \\
		Ability to stay anonymous & Anonymity can be abused \\
		Huge pool of information & Risk of data leakage or doxing \\
		Great diversity in people & Forming of political bubbles \\
		Easy ways to communicate and share personal life & Addictive \linebreak\rightarrow  mental health issues \\
		New jobs & High emissions through servers using lots of electrical power \\
		Freedom of speech & No general protection from hate speech\\
		Quick access to entertainment & Influencers influencing decision making based on sponsors \\ \hline
	\end{tabularx}
	\caption{Opportunities and challenges of the Web 2.0}
	\label{tab:my_label}
\end{table}

\subsection{English as a global language}
	\label{ssec:media@english}

\section{Crooked Letter, Crooked Letter}
	\label{sec:crookedl}
\subsection{Ambiguity of belonging}
	\label{ssec:crookedl@ambofbelonging}

\end{document}
